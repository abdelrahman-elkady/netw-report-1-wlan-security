\chapter{Authentication and Association}
\label{chap:auth_association_cycle}

\section{Authentication}
\label{sec:authentication}
For a node to connect to an \ac{ap} in the network, it needs to authenticate with the access
point. After checking the \ac{ssid} of the desired \ac{ap} the authentication process is established. The discovery
of nearby networks is done throughout a probe process, this is accomplished by sending probe request frames
to the nearby networks, which mainly holds two pieces of information (\ac{ssid} and data rate).\cite{roshan_leary_2004}
If one network is compatible with this request, it replies with a probe response frame containing the information for
the node to connect to the network.\cite{gast_2005}

\newParagraph
Authentication in 802.11 is used to determine which device is allowed to use the network. This is done by sending
a management frame which is a part of 802.11 standard. Those management frames contains the authentication algorithm
indicator, which determines the used authentication systems used in 802.11 which are Open System Authentication and
Shared Key Authentication.\cite{gast_2005}

\subsection{Open System Authentication}
\label{sub:open_system_auth}
In open system authentication, the authentication process is done without verifying the identity of the connecting node,
that means that the \ac{ap} is just accepting a frame from the node to connect to it. Open System Authentication is
often used with \ac{wep} to ensure privacy, however, \ac{wep} is active in Open System Authentication after the authentication
and association processes are done.\cite{cwsp-205_2016}

\subsection{Shared Key Authentication}
\label{sub:shared_key_auth}
Shared Key Authentication uses \ac{wep} to restrict access to nodes that have a pre-defined key shared between
the \ac{ap}(s) and the nodes. The Authentication process requires four-way handshake. At the beginning a normal
authentication request is sent to the \ac{ap}. Then the client node uses its key (which is shared between the nodes
and the \ac{ap}) to encrypt a payload text in the request. The \ac{ap} receives the request and the payload and
decrypts it, if the decryption showed a valid payload, a success authentication response is sent, otherwise, the
response is sent indicating a failed authentication process.\cite{cwsp-205_2016}


\section{Association}
\label{sec:association}
The Association process is established after a successful authentication between a client node and the \ac{ap}.
In this process the client node requests to join the \ac{bss} to send and receive data with the network. Initially
an association request frame is sent to the \ac{ap}. The \ac{ap} then sends an association response indicating whether
it is allowed for the client node to join the \ac{bss} or not. If the \ac{ap} responded with a success, this response
will contain some information for the association process. Mainly the response frame contains an \ac{aid} which is a unique
identifier generated to differentiate between different nodes in the \ac{bss}.\cite{cwsp-205_2016}


\section{Association and Authentication States}
\label{sec:association_authentication_states}
The \ac{ap} keeps track of authentication and association state for the client nodes. That results in three
possible states for each node:
\begin{enumerate}
  \item Not authenticated nor associated
  \item Authenticated but not associated
  \item Authenticated and associated
\end{enumerate}

Initially the default state for a new node joining the \ac{bss} is state 1. In state 1 and 2 no data transfer could
happen between the client node and the network. As shown in Figure \ref{fig:auth_association_states},
there are different frame classes that can be used in each state. Class 1 Frames are the control frames, they can be used
in all the three states as they handles the authentication and probing requests. Class 2 Frames are allowed in state 2 and 3,
those frames are mainly the association request and response frames. Class 3 frames are used when the node is authenticated
and associated, those frames allows communicating with the distribution system in the \ac{bss}.\cite{gast_2005}

\includefig{0.85}{association-authentication-cycle.png}{Authentication and association states}{fig:auth_association_states}
